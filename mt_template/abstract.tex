\chapter*{Abstract}

%This thesis addresses the development of a novel sample thesis. We analyze the requirements of a general template, as it can be used with the \LaTeX\ text processing system. (And so on\dots) The abstract should not exceed half a page in size!

The goal of this thesis was to create a tool for computing and displaying wind trajectories. We would then use it to visualize several test cases from meteorological data.% Said data deals with F�hn, a phenomenon that is associated with warm and dry winds in the northern parts of the alps, also bad weather on the south side.

While the focus lay on visualization and analysis of results initially, it later shifted towards reproducing trajectories from a similar tool and improving on that. That worked in the sense that we can obtain trajectories that are practically the same. We can also do it faster and with fewer errors. The visualization part ended up as more of a debugging tool than a user-friendly application.

% "over the alpine region". LAGRANTO is used as a template for computing trajectories and VTK for drawing the results. While the initial focus was on the visualization, it shifted from making the basics work to comparing results with LAGRANTO over the duration of three months.

\cleardoublepage
\chapter*{Zusammenfassung}

Das Ziel dieser Arbeit war es, ein Tool zur Berechnung und Darstellung von Windtrajektorien zu erstellen. Es w�rde dann zum Visualisieren etlicher Testf�lle aus meteorologischen Daten benutzt werden.% Die Daten besch�ftigen sich mit dem F�hn, einem Ph�nomen das mit warmen und trockenen Winden in den n�rdlichen Teilen der Alpen verbunden wird. Ausserdem schlechtes Wetter auf der S�dseite.

Obwohl der Schwerpunkt urspr�nglich auf dem Visualisieren und Analysieren von Resultaten lag, �nderte er sich sp�ter und es wurde mehr darauf geachtet, dass die Trajektorien aus einem �hnlichen Programm reproduziert und verbessert werden k�nnen. Dies funktionierte in der Hinsicht, dass wir praktische identische Trajektorien produzieren k�nnen. Dabei k�nnen wir es auch schneller und mit weniger Fehlern. Der Visualisierungsteil wurde daher eher zu einem Debuggingwerkzeug als zu einer benutzerfreundlichen Anwendung.

%