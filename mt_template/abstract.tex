\chapter*{Abstract}

% to say here: goal (compute and visualize trajectories)
%   short intro & background: working with meteorologists for SCIENCE
%   short method: copying Lagranto
%   short results: yeah, we can do it
%   short conclusion: ehh, the visualization part is not very impressive. can u fix it kthxbye
%also maybe time was a bit too short

The goal of this thesis was to create a tool for computing and displaying wind trajectories. 
This was done in collaboration with meteorologists from the Institute of Atmospheric and Climate Science. Our main contribution is a C++ analogue of the Lagrangian Analyis Tool LAGRANTO.
%Our main contribution is a C++ replacement for an older Fortran thing
% combine: Working together with meteorologists from D-USYS we made a C++ analogue of the Lagrangian Analyis Tool LAGRANTO. 

%We compared results by visualizing various trajectories obtained from both programs using the same input data.
To evaluate our results we looked at trajectories obtained from both programs using the same input data.
We found our implementation to have better speed and accuracy, but we lack some of the features.

%The goal of this thesis was to create a tool for computing and displaying wind trajectories. It is then used to visualize several test cases from meteorological data.% Said data deals with F�hn, a phenomenon that is associated with warm and dry winds in the northern parts of the alps, also bad weather on the south side.

While the focus lay on visualization and analysis of results initially, it later shifted towards reproducing trajectories from the template and improving on that. This worked in the sense that we can obtain trajectories that are practically the same. The visualization became more of a debugging tool than a user-friendly application in the end.

\cleardoublepage
\chapter*{Zusammenfassung}

Das Ziel dieser Arbeit war es, ein Tool zur Berechnung und Darstellung von Windtrajektorien zu erstellen. Dies wurde in Zusammenarbeit mit Meteorologen vom Institut f�r Atmosph�re und Klima getan. Unsere Hauptleistung ist ein in C++ geschriebenes Programm mit �hnlichen Funktionen wie das Lagrangian Analysis Tool LAGRANTO.

Um unsere Resultate auszuwerten haben wir Trajektorien angeschaut, die von beiden Programmen mit den gleichen Eingabedaten ausgerechnet wurden. Dabei fanden wir heraus, dass unsere Implementierung besser ist in Sachen Geschwindigkeit und Genauigkeit, daf�r fehlen einige der Features.

%Das Ziel dieser Arbeit war es, ein Tool zur Berechnung und Darstellung von Windtrajektorien zu erstellen. Dann wird es zum Visualisieren etlicher Testf�lle aus meteorologischen Daten benutzt.% Die Daten besch�ftigen sich mit dem F�hn, einem Ph�nomen das mit warmen und trockenen Winden in den n�rdlichen Teilen der Alpen verbunden wird. Ausserdem schlechtes Wetter auf der S�dseite.

Obwohl der Schwerpunkt urspr�nglich auf dem Visualisieren und Analysieren von Resultaten lag, �nderte er sich sp�ter, und es wurde mehr darauf geachtet, dass die Trajektorien aus der Vorlage reproduziert und verbessert werden k�nnen. Dies funktionierte in der Hinsicht, dass wir praktisch identische Trajektorien produzieren k�nnen. Der Visualisierungsteil wurde daher eher zu einem Debuggingwerkzeug als zu einer benutzerfreundlichen Anwendung.

%